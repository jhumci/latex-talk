\documentclass{scrartcl}

% Kodierung dieser Datei angeben
\usepackage[utf8]{inputenc}

% Schönere Schriftart laden
\usepackage[T1]{fontenc}
\usepackage{lmodern}

% Deutsche Silbentrennung verwenden
\usepackage[ngerman]{babel}

% Bessere Unterstützung für PDF-Features
\usepackage[breaklinks=true]{hyperref}

\KOMAoptions{%
  % Absätze durch Abstände
  parskip=full,%
  % Satzspiegel berechnen lassen
  DIV=calc,%
  % Modernen, offenen Stil für Literaturverzeichnisse aktivieren
  bibliography=openstyle,%
}

\begin{document}
  In \cite{Knuth} wird das Satzsystem \TeX{}
  vom Autor des Systems
  vorgestellt. Jedes Zeichen hat dabei einen Category Code
  (vergleiche dazu \cite[S.~28~ff.]{Eijkhout}).

  \bibliographystyle{alphadin}
  \bibliography{literatur}
\end{document}