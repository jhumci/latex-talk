%!TEX root = latex.tex

\begin{frame}{Modulare Dokumente}
  \tikzset{
    every node/.style={
      inner sep=1pt,
      on grid,
      label position=below,
      auto
    },
    shorten >=1em,
    shorten <=1em,
    node distance=7em and 5em
  }
  \begin{center}
    \begin{tikzpicture}
      \node[inner sep=10pt] (large) {\tikz\node[examplecolor,label=\texttt{\ \ main}] {\only<presentation:1| article:0>{\icon[2]{TEX}}\only<presentation:2| article:0>{\icon[10]{TEX}}\only<presentation:3| article:0>{\icon[20]{TEX}}\only<4->{\icon[30]{TEX}}};};
      \onslide<5->
      \path[alertedcolor, line width=3pt, line cap=round, shorten >=0pt,shorten <=0pt]
        (large.north east) edge (large.south west)
        (large.north west) edge (large.south east);
    \end{tikzpicture}
    \qquad
    \only<6->{%
      \begin{tikzpicture}
        \node[examplecolor,label=below:\texttt{\ \ main}] (main) {\icon[1]{TEX}};

        \uncover<7->{%
          \node[maincolor,label=\texttt{\ \ styles}, above left=of main] (config) {\icon[2]{TEX}};
          \path[very thick]
            (config.east) edge[->] node[pos=.6, swap]
              {\texttt{\color{texcs}\bfseries\textbackslash input}} (main);
        }

        \uncover<8->{%
          \node[maincolor,label=\texttt{\ \ intro}, above right=of main] (chap1) {\icon[3]{TEX}};
          \path[very thick]
            (chap1.west) edge[->] node[pos=.6]
              {\texttt{\color{texcs}\bfseries\textbackslash include}} (main);
        }

        \uncover<9->{%
          \node[maincolor,label=\texttt{\ \ methods}, below right=of main] (chap2) {\icon[5]{TEX}};
          \path[very thick]
            (chap2.west) edge[->,shorten >=1.6em] node[pos=.5, swap]
              {\texttt{\color{texcs}\bfseries\textbackslash include}} (main);
        }

        \uncover<10->{%
          \node[maincolor,label=\texttt{\ \ summary}, below left=of main] (chap3) {\icon[4]{TEX}};
          \path[very thick]
            (chap3.east) edge[->,shorten >=1.6em] node[pos=.5]
              {\texttt{\color{texcs}\bfseries\textbackslash include}} (main);
        }
      \end{tikzpicture}
    }
  \end{center}
\end{frame}

\begin{frame}[fragile]{Modulare Struktur}
  \begin{columns}
    \column{.4\textwidth}
    \begin{tikzpicture}[
        node distance=0,
        grow via three points={one child at (0,-0.7) and two children at (0,-0.7) and (0,-1.4)},
        edge from parent path={([xshift=6pt]\tikzparentnode.south west) |- (\tikzchildnode.west)},
        every node/.style={thick,anchor=west,xshift=-3mm,font=\ttfamily,text width=4em},
        every child/.style={thick,draw=black},
        dir/.style={draw=maincolor,fill=maincolor!10},
        optional/.style={dashed}
      ]
      \node [dir] {thesis}
        child { node [dir] {inc}
          child { node {title}}
          child { node {styles}}
        }
        child [missing] {}
        child [missing] {}
        child { node [dir] {content}
          child { node {intro}}
          child { node (methods) {methods}}
          child { node (summary) {summary}}
        }
        child [missing] {}
        child [missing] {}
        child [missing] {}
        child { node {main}};
        \only<2>\path[alertedcolor, line width=1pt, line cap=rounded]
          (methods.south west) edge (methods.north east);
        \only<2>\path[alertedcolor, line width=1pt, line cap=rounded]
          (summary.south west) edge (summary.north east);
      \end{tikzpicture}
    \column{0.6\textwidth}
    \begin{lstlisting}[gobble=6,escapechar=-]
      \documentclass{scrbook}
      \input{inc/styles}
      -\alt<2>{\textcolor{texcs}{\bfseries\textbackslash includeonly}\textcolor{red}{\bfseries\{}content/intro\textcolor{red}{\bfseries\}}}{\textcolor{comment}{\itshape\% \textbackslash includeonly}}-
      \begin{document}
        \frontmatter
        %!TEX root = thesis.tex

\begin{titlepage}
  \thispagestyle{empty}

  \vskip1cm

  \pgfimage[height=2.5cm]{uni-logo-example\imagesuffix}
  
  \vskip2.5cm
  
  \LARGE
  
  \textbf{\sffamily\color{maincolor}Über Gummibärchen}

  \textit{On Gummy Bears}

  \normalfont\normalsize

  \vskip2em
  
  \textbf{\sffamily\color{maincolor}Masterarbeit}

  im Rahmen des Studiengangs \\
  \textbf{\sffamily\color{maincolor}Informatik} \\
  der Universität zum Beispiel

  \vskip1em

  vorgelegt von \\
  \textbf{\sffamily\color{maincolor}Max Mustermann}

  \vskip1em
  
  ausgegeben und betreut von \\
  \textbf{\sffamily\color{maincolor}Prof. Dr. Erika Musterfrau}

  \vskip1em

  mit Unterstützung von\\
  Lieschen Müller

  \vskip1em

  Die Arbeit ist im Rahmen einer Tätigkeit bei der Firma Muster GmbH entstanden.


  \vfill

  Musterhausen, den \duedate
\end{titlepage}

        \tableofcontents

        \mainmatter
        \include{content/intro}
        \include{content/methods}
        %!TEX root = basic.tex

\begin{frame}[label=basics-summary]{Zusammenfassung}
  \begin{enumerate}
    \item Das \alert{\LaTeX-Dokument} enthält \alert{Inhalt und Struktur}.
    \item \LaTeX\ setzt ein druckfertiges \alert{PDF-Dokument} und kümmert sich dabei um die \alert{gute Form}.
    \item Es ist schwierig, \alert{neue Layouts} zu erzeugen.
    \item Ein \LaTeX-Dokument besteht aus \alert{Dokumentenklasse}, \alert{Präambel} und \alert{Dokumentenkörper}.
    \item Wir haben \alert{Auszeichnungen}, \alert{Formelsatz}, \alert{Listen}, \alert{Tabellen}, \alert{Abbildungen}, \alert{Verzeichnisse} und \alert{Verweise} kennen gelernt.
  \end{enumerate}
\end{frame}

\begin{frame}[fragile]{Zum Weiterlesen}
  \begin{mybib}
    \bibitem{Wiki}
      Wikibooks contributors.
      \newblock \emph{\LaTeX\ Wikibook},
      \newblock \alt<presentation>{\href{http://en.wikibooks.org/wiki/LaTeX}{\texttt{en.wikibooks.org/LaTeX}}}{\url{http://en.wikibooks.org/wiki/LaTeX}}, November 2014
    \bibitem{basics_Kohm}
      Markus Kohm, Jens-Uwe-Morawski.
      \newblock \emph{KOMA-Script},
      \newblock \alt<presentation>{\href{http://mirrors.ctan.org/macros/latex/contrib/koma-script/doc/scrguide.pdf}{\texttt{scrguide.pdf}}}{\url{http://mirrors.ctan.org/macros/latex/contrib/koma-script/doc/scrguide.pdf}}, Dezember 2013.
  \end{mybib}
\end{frame}

\begin{frame}[fragile]{Zum weiteren Weiterlesen}
  \begin{mybib}
    \bibitem{basics_Kopka}
      Helmut Kopka.
      \newblock \emph{\LaTeX, Band 1: Einführung},
      \newblock Addison-Wesley, März 2002.
    \bibitem{basics_Braune}
      Klaus Braune, Joachim und Marion Lammarsch.
      \newblock \emph{\LaTeX: Basissystem, Layout, Formelsatz},
      \newblock Addison-Wesley, Mai 2006.
    \bibitem{Struckmann}
      Werner Struckmann.
      \newblock \emph{Einige typographische Grundregeln und ihre Umsetzung in \LaTeX},
      \newblock \alt<presentation>{\href{http://www2.informatik.hu-berlin.de/sv/lehre/typographie.pdf}{\texttt{typographie.pdf}}}{\url{http://www2.informatik.hu-berlin.de/sv/lehre/typographie.pdf}}, September 2007.
  \end{mybib}
\end{frame}


      \end{document}
    \end{lstlisting}
  \end{columns}
\end{frame}

\begin{frame}[fragile]{Befehle für modulare Struktur}
  \begin{tabular}{rL{6cm}}
    \lstinline-\input- & Inhalt der Datei einfügen. \\[1ex]
    \lstinline-\include- & Inhalt der Datei einfügen und \newline
      Seitenumbrüche davor und dahinter.\\[1ex]
    \lstinline-\includeonly- & Liste der von \lstinline-\include- \newline
      berücksichtigten Dateien.\\
  \end{tabular}
\end{frame}

\begin{frame}[fragile]{Abschnitte langer Dokumente (\lstinline-scrbook-)}
  \begin{lstlisting}[gobble=4]
    \begin{document}
      \frontmatter % Vorspann
      \begin{titlepage} ... \end{titlepage}
      \tableofcontents

      \mainmatter % Hauptteil
      \chapter{Einleitung}

      \appendix % Anhang
      \chapter{Glossar}
      
      \backmatter % Nachspann
      \listoffigures
    \end{document}
  \end{lstlisting}
\end{frame}

\begin{frame}[fragile]{Gestaltung der Abschnitte}
  \begin{tabbing}
    \hskip8cm \= \kill

    \textcolor{texcs}{\ttfamily\bfseries\textbackslash frontmatter} Vorspann
        \> \textcolor{examplecolor}{\bfseries\sffamily Vorwort}\\
      \strut\ \textcolor{maincolor}{--} keine Kapitelnummern\\
      \strut\ \textcolor{maincolor}{--} römische Seitennummern
        \> \textcolor{examplecolor}{\rmfamily \qquad -- iv --} \\[3ex]

    \pause

    \textcolor{texcs}{\ttfamily\bfseries\textbackslash mainmatter} Hauptteil
        \> \textcolor{examplecolor}{\bfseries\sffamily 3. Konzept}\\
      \strut\ \textcolor{maincolor}{--} arabische Kapitelnummern\\
      \strut\ \textcolor{maincolor}{--} \alert{neue} arabische Seitennummern
        \> \textcolor{examplecolor}{\rmfamily \qquad -- 5 --} \\[3ex]

    \pause

    \textcolor{texcs}{\ttfamily\bfseries\textbackslash appendix} Anhang
        \> \textcolor{examplecolor}{\bfseries\sffamily A. Anhang}\\
      \strut\ \textcolor{maincolor}{--} \alert{neue} alphabetische Kapitelnummern\\
      \strut\ \textcolor{maincolor}{--} arabische Seitennummern
        \> \textcolor{examplecolor}{\rmfamily \qquad -- 126 --} \\[3ex]

    \pause

    \textcolor{texcs}{\ttfamily\bfseries\textbackslash backmatter} Nachspann
        \> \textcolor{examplecolor}{\bfseries\sffamily Literatur}\\
      \strut\ \textcolor{maincolor}{--} keine Kapitelnummern\\
      \strut\ \textcolor{maincolor}{--} arabische Seitennummern
        \> \textcolor{examplecolor}{\rmfamily \qquad -- 135 --}
  \end{tabbing}
\end{frame}

